
% Default to the notebook output style

    


% Inherit from the specified cell style.




    
\documentclass[11pt]{article}

    
    
    \usepackage[T1]{fontenc}
    % Nicer default font (+ math font) than Computer Modern for most use cases
    \usepackage{mathpazo}

    % Basic figure setup, for now with no caption control since it's done
    % automatically by Pandoc (which extracts ![](path) syntax from Markdown).
    \usepackage{graphicx}
    % We will generate all images so they have a width \maxwidth. This means
    % that they will get their normal width if they fit onto the page, but
    % are scaled down if they would overflow the margins.
    \makeatletter
    \def\maxwidth{\ifdim\Gin@nat@width>\linewidth\linewidth
    \else\Gin@nat@width\fi}
    \makeatother
    \let\Oldincludegraphics\includegraphics
    % Set max figure width to be 80% of text width, for now hardcoded.
    \renewcommand{\includegraphics}[1]{\Oldincludegraphics[width=.8\maxwidth]{#1}}
    % Ensure that by default, figures have no caption (until we provide a
    % proper Figure object with a Caption API and a way to capture that
    % in the conversion process - todo).
    \usepackage{caption}
    \DeclareCaptionLabelFormat{nolabel}{}
    \captionsetup{labelformat=nolabel}

    \usepackage{adjustbox} % Used to constrain images to a maximum size 
    \usepackage{xcolor} % Allow colors to be defined
    \usepackage{enumerate} % Needed for markdown enumerations to work
    \usepackage{geometry} % Used to adjust the document margins
    \usepackage{amsmath} % Equations
    \usepackage{amssymb} % Equations
    \usepackage{textcomp} % defines textquotesingle
    % Hack from http://tex.stackexchange.com/a/47451/13684:
    \AtBeginDocument{%
        \def\PYZsq{\textquotesingle}% Upright quotes in Pygmentized code
    }
    \usepackage{upquote} % Upright quotes for verbatim code
    \usepackage{eurosym} % defines \euro
    \usepackage[mathletters]{ucs} % Extended unicode (utf-8) support
    \usepackage[utf8x]{inputenc} % Allow utf-8 characters in the tex document
    \usepackage{fancyvrb} % verbatim replacement that allows latex
    \usepackage{grffile} % extends the file name processing of package graphics 
                         % to support a larger range 
    % The hyperref package gives us a pdf with properly built
    % internal navigation ('pdf bookmarks' for the table of contents,
    % internal cross-reference links, web links for URLs, etc.)
    \usepackage{hyperref}
    \usepackage{longtable} % longtable support required by pandoc >1.10
    \usepackage{booktabs}  % table support for pandoc > 1.12.2
    \usepackage[inline]{enumitem} % IRkernel/repr support (it uses the enumerate* environment)
    \usepackage[normalem]{ulem} % ulem is needed to support strikethroughs (\sout)
                                % normalem makes italics be italics, not underlines
    

    
    
    % Colors for the hyperref package
    \definecolor{urlcolor}{rgb}{0,.145,.698}
    \definecolor{linkcolor}{rgb}{.71,0.21,0.01}
    \definecolor{citecolor}{rgb}{.12,.54,.11}

    % ANSI colors
    \definecolor{ansi-black}{HTML}{3E424D}
    \definecolor{ansi-black-intense}{HTML}{282C36}
    \definecolor{ansi-red}{HTML}{E75C58}
    \definecolor{ansi-red-intense}{HTML}{B22B31}
    \definecolor{ansi-green}{HTML}{00A250}
    \definecolor{ansi-green-intense}{HTML}{007427}
    \definecolor{ansi-yellow}{HTML}{DDB62B}
    \definecolor{ansi-yellow-intense}{HTML}{B27D12}
    \definecolor{ansi-blue}{HTML}{208FFB}
    \definecolor{ansi-blue-intense}{HTML}{0065CA}
    \definecolor{ansi-magenta}{HTML}{D160C4}
    \definecolor{ansi-magenta-intense}{HTML}{A03196}
    \definecolor{ansi-cyan}{HTML}{60C6C8}
    \definecolor{ansi-cyan-intense}{HTML}{258F8F}
    \definecolor{ansi-white}{HTML}{C5C1B4}
    \definecolor{ansi-white-intense}{HTML}{A1A6B2}

    % commands and environments needed by pandoc snippets
    % extracted from the output of `pandoc -s`
    \providecommand{\tightlist}{%
      \setlength{\itemsep}{0pt}\setlength{\parskip}{0pt}}
    \DefineVerbatimEnvironment{Highlighting}{Verbatim}{commandchars=\\\{\}}
    % Add ',fontsize=\small' for more characters per line
    \newenvironment{Shaded}{}{}
    \newcommand{\KeywordTok}[1]{\textcolor[rgb]{0.00,0.44,0.13}{\textbf{{#1}}}}
    \newcommand{\DataTypeTok}[1]{\textcolor[rgb]{0.56,0.13,0.00}{{#1}}}
    \newcommand{\DecValTok}[1]{\textcolor[rgb]{0.25,0.63,0.44}{{#1}}}
    \newcommand{\BaseNTok}[1]{\textcolor[rgb]{0.25,0.63,0.44}{{#1}}}
    \newcommand{\FloatTok}[1]{\textcolor[rgb]{0.25,0.63,0.44}{{#1}}}
    \newcommand{\CharTok}[1]{\textcolor[rgb]{0.25,0.44,0.63}{{#1}}}
    \newcommand{\StringTok}[1]{\textcolor[rgb]{0.25,0.44,0.63}{{#1}}}
    \newcommand{\CommentTok}[1]{\textcolor[rgb]{0.38,0.63,0.69}{\textit{{#1}}}}
    \newcommand{\OtherTok}[1]{\textcolor[rgb]{0.00,0.44,0.13}{{#1}}}
    \newcommand{\AlertTok}[1]{\textcolor[rgb]{1.00,0.00,0.00}{\textbf{{#1}}}}
    \newcommand{\FunctionTok}[1]{\textcolor[rgb]{0.02,0.16,0.49}{{#1}}}
    \newcommand{\RegionMarkerTok}[1]{{#1}}
    \newcommand{\ErrorTok}[1]{\textcolor[rgb]{1.00,0.00,0.00}{\textbf{{#1}}}}
    \newcommand{\NormalTok}[1]{{#1}}
    
    % Additional commands for more recent versions of Pandoc
    \newcommand{\ConstantTok}[1]{\textcolor[rgb]{0.53,0.00,0.00}{{#1}}}
    \newcommand{\SpecialCharTok}[1]{\textcolor[rgb]{0.25,0.44,0.63}{{#1}}}
    \newcommand{\VerbatimStringTok}[1]{\textcolor[rgb]{0.25,0.44,0.63}{{#1}}}
    \newcommand{\SpecialStringTok}[1]{\textcolor[rgb]{0.73,0.40,0.53}{{#1}}}
    \newcommand{\ImportTok}[1]{{#1}}
    \newcommand{\DocumentationTok}[1]{\textcolor[rgb]{0.73,0.13,0.13}{\textit{{#1}}}}
    \newcommand{\AnnotationTok}[1]{\textcolor[rgb]{0.38,0.63,0.69}{\textbf{\textit{{#1}}}}}
    \newcommand{\CommentVarTok}[1]{\textcolor[rgb]{0.38,0.63,0.69}{\textbf{\textit{{#1}}}}}
    \newcommand{\VariableTok}[1]{\textcolor[rgb]{0.10,0.09,0.49}{{#1}}}
    \newcommand{\ControlFlowTok}[1]{\textcolor[rgb]{0.00,0.44,0.13}{\textbf{{#1}}}}
    \newcommand{\OperatorTok}[1]{\textcolor[rgb]{0.40,0.40,0.40}{{#1}}}
    \newcommand{\BuiltInTok}[1]{{#1}}
    \newcommand{\ExtensionTok}[1]{{#1}}
    \newcommand{\PreprocessorTok}[1]{\textcolor[rgb]{0.74,0.48,0.00}{{#1}}}
    \newcommand{\AttributeTok}[1]{\textcolor[rgb]{0.49,0.56,0.16}{{#1}}}
    \newcommand{\InformationTok}[1]{\textcolor[rgb]{0.38,0.63,0.69}{\textbf{\textit{{#1}}}}}
    \newcommand{\WarningTok}[1]{\textcolor[rgb]{0.38,0.63,0.69}{\textbf{\textit{{#1}}}}}
    
    
    % Define a nice break command that doesn't care if a line doesn't already
    % exist.
    \def\br{\hspace*{\fill} \\* }
    % Math Jax compatability definitions
    \def\gt{>}
    \def\lt{<}
    % Document parameters
    \title{11160940000076\_MAFTUH MASHURI\_TUGAS III\_DATAWAREHOUSE}
    
    
    

    % Pygments definitions
    
\makeatletter
\def\PY@reset{\let\PY@it=\relax \let\PY@bf=\relax%
    \let\PY@ul=\relax \let\PY@tc=\relax%
    \let\PY@bc=\relax \let\PY@ff=\relax}
\def\PY@tok#1{\csname PY@tok@#1\endcsname}
\def\PY@toks#1+{\ifx\relax#1\empty\else%
    \PY@tok{#1}\expandafter\PY@toks\fi}
\def\PY@do#1{\PY@bc{\PY@tc{\PY@ul{%
    \PY@it{\PY@bf{\PY@ff{#1}}}}}}}
\def\PY#1#2{\PY@reset\PY@toks#1+\relax+\PY@do{#2}}

\expandafter\def\csname PY@tok@w\endcsname{\def\PY@tc##1{\textcolor[rgb]{0.73,0.73,0.73}{##1}}}
\expandafter\def\csname PY@tok@c\endcsname{\let\PY@it=\textit\def\PY@tc##1{\textcolor[rgb]{0.25,0.50,0.50}{##1}}}
\expandafter\def\csname PY@tok@cp\endcsname{\def\PY@tc##1{\textcolor[rgb]{0.74,0.48,0.00}{##1}}}
\expandafter\def\csname PY@tok@k\endcsname{\let\PY@bf=\textbf\def\PY@tc##1{\textcolor[rgb]{0.00,0.50,0.00}{##1}}}
\expandafter\def\csname PY@tok@kp\endcsname{\def\PY@tc##1{\textcolor[rgb]{0.00,0.50,0.00}{##1}}}
\expandafter\def\csname PY@tok@kt\endcsname{\def\PY@tc##1{\textcolor[rgb]{0.69,0.00,0.25}{##1}}}
\expandafter\def\csname PY@tok@o\endcsname{\def\PY@tc##1{\textcolor[rgb]{0.40,0.40,0.40}{##1}}}
\expandafter\def\csname PY@tok@ow\endcsname{\let\PY@bf=\textbf\def\PY@tc##1{\textcolor[rgb]{0.67,0.13,1.00}{##1}}}
\expandafter\def\csname PY@tok@nb\endcsname{\def\PY@tc##1{\textcolor[rgb]{0.00,0.50,0.00}{##1}}}
\expandafter\def\csname PY@tok@nf\endcsname{\def\PY@tc##1{\textcolor[rgb]{0.00,0.00,1.00}{##1}}}
\expandafter\def\csname PY@tok@nc\endcsname{\let\PY@bf=\textbf\def\PY@tc##1{\textcolor[rgb]{0.00,0.00,1.00}{##1}}}
\expandafter\def\csname PY@tok@nn\endcsname{\let\PY@bf=\textbf\def\PY@tc##1{\textcolor[rgb]{0.00,0.00,1.00}{##1}}}
\expandafter\def\csname PY@tok@ne\endcsname{\let\PY@bf=\textbf\def\PY@tc##1{\textcolor[rgb]{0.82,0.25,0.23}{##1}}}
\expandafter\def\csname PY@tok@nv\endcsname{\def\PY@tc##1{\textcolor[rgb]{0.10,0.09,0.49}{##1}}}
\expandafter\def\csname PY@tok@no\endcsname{\def\PY@tc##1{\textcolor[rgb]{0.53,0.00,0.00}{##1}}}
\expandafter\def\csname PY@tok@nl\endcsname{\def\PY@tc##1{\textcolor[rgb]{0.63,0.63,0.00}{##1}}}
\expandafter\def\csname PY@tok@ni\endcsname{\let\PY@bf=\textbf\def\PY@tc##1{\textcolor[rgb]{0.60,0.60,0.60}{##1}}}
\expandafter\def\csname PY@tok@na\endcsname{\def\PY@tc##1{\textcolor[rgb]{0.49,0.56,0.16}{##1}}}
\expandafter\def\csname PY@tok@nt\endcsname{\let\PY@bf=\textbf\def\PY@tc##1{\textcolor[rgb]{0.00,0.50,0.00}{##1}}}
\expandafter\def\csname PY@tok@nd\endcsname{\def\PY@tc##1{\textcolor[rgb]{0.67,0.13,1.00}{##1}}}
\expandafter\def\csname PY@tok@s\endcsname{\def\PY@tc##1{\textcolor[rgb]{0.73,0.13,0.13}{##1}}}
\expandafter\def\csname PY@tok@sd\endcsname{\let\PY@it=\textit\def\PY@tc##1{\textcolor[rgb]{0.73,0.13,0.13}{##1}}}
\expandafter\def\csname PY@tok@si\endcsname{\let\PY@bf=\textbf\def\PY@tc##1{\textcolor[rgb]{0.73,0.40,0.53}{##1}}}
\expandafter\def\csname PY@tok@se\endcsname{\let\PY@bf=\textbf\def\PY@tc##1{\textcolor[rgb]{0.73,0.40,0.13}{##1}}}
\expandafter\def\csname PY@tok@sr\endcsname{\def\PY@tc##1{\textcolor[rgb]{0.73,0.40,0.53}{##1}}}
\expandafter\def\csname PY@tok@ss\endcsname{\def\PY@tc##1{\textcolor[rgb]{0.10,0.09,0.49}{##1}}}
\expandafter\def\csname PY@tok@sx\endcsname{\def\PY@tc##1{\textcolor[rgb]{0.00,0.50,0.00}{##1}}}
\expandafter\def\csname PY@tok@m\endcsname{\def\PY@tc##1{\textcolor[rgb]{0.40,0.40,0.40}{##1}}}
\expandafter\def\csname PY@tok@gh\endcsname{\let\PY@bf=\textbf\def\PY@tc##1{\textcolor[rgb]{0.00,0.00,0.50}{##1}}}
\expandafter\def\csname PY@tok@gu\endcsname{\let\PY@bf=\textbf\def\PY@tc##1{\textcolor[rgb]{0.50,0.00,0.50}{##1}}}
\expandafter\def\csname PY@tok@gd\endcsname{\def\PY@tc##1{\textcolor[rgb]{0.63,0.00,0.00}{##1}}}
\expandafter\def\csname PY@tok@gi\endcsname{\def\PY@tc##1{\textcolor[rgb]{0.00,0.63,0.00}{##1}}}
\expandafter\def\csname PY@tok@gr\endcsname{\def\PY@tc##1{\textcolor[rgb]{1.00,0.00,0.00}{##1}}}
\expandafter\def\csname PY@tok@ge\endcsname{\let\PY@it=\textit}
\expandafter\def\csname PY@tok@gs\endcsname{\let\PY@bf=\textbf}
\expandafter\def\csname PY@tok@gp\endcsname{\let\PY@bf=\textbf\def\PY@tc##1{\textcolor[rgb]{0.00,0.00,0.50}{##1}}}
\expandafter\def\csname PY@tok@go\endcsname{\def\PY@tc##1{\textcolor[rgb]{0.53,0.53,0.53}{##1}}}
\expandafter\def\csname PY@tok@gt\endcsname{\def\PY@tc##1{\textcolor[rgb]{0.00,0.27,0.87}{##1}}}
\expandafter\def\csname PY@tok@err\endcsname{\def\PY@bc##1{\setlength{\fboxsep}{0pt}\fcolorbox[rgb]{1.00,0.00,0.00}{1,1,1}{\strut ##1}}}
\expandafter\def\csname PY@tok@kc\endcsname{\let\PY@bf=\textbf\def\PY@tc##1{\textcolor[rgb]{0.00,0.50,0.00}{##1}}}
\expandafter\def\csname PY@tok@kd\endcsname{\let\PY@bf=\textbf\def\PY@tc##1{\textcolor[rgb]{0.00,0.50,0.00}{##1}}}
\expandafter\def\csname PY@tok@kn\endcsname{\let\PY@bf=\textbf\def\PY@tc##1{\textcolor[rgb]{0.00,0.50,0.00}{##1}}}
\expandafter\def\csname PY@tok@kr\endcsname{\let\PY@bf=\textbf\def\PY@tc##1{\textcolor[rgb]{0.00,0.50,0.00}{##1}}}
\expandafter\def\csname PY@tok@bp\endcsname{\def\PY@tc##1{\textcolor[rgb]{0.00,0.50,0.00}{##1}}}
\expandafter\def\csname PY@tok@fm\endcsname{\def\PY@tc##1{\textcolor[rgb]{0.00,0.00,1.00}{##1}}}
\expandafter\def\csname PY@tok@vc\endcsname{\def\PY@tc##1{\textcolor[rgb]{0.10,0.09,0.49}{##1}}}
\expandafter\def\csname PY@tok@vg\endcsname{\def\PY@tc##1{\textcolor[rgb]{0.10,0.09,0.49}{##1}}}
\expandafter\def\csname PY@tok@vi\endcsname{\def\PY@tc##1{\textcolor[rgb]{0.10,0.09,0.49}{##1}}}
\expandafter\def\csname PY@tok@vm\endcsname{\def\PY@tc##1{\textcolor[rgb]{0.10,0.09,0.49}{##1}}}
\expandafter\def\csname PY@tok@sa\endcsname{\def\PY@tc##1{\textcolor[rgb]{0.73,0.13,0.13}{##1}}}
\expandafter\def\csname PY@tok@sb\endcsname{\def\PY@tc##1{\textcolor[rgb]{0.73,0.13,0.13}{##1}}}
\expandafter\def\csname PY@tok@sc\endcsname{\def\PY@tc##1{\textcolor[rgb]{0.73,0.13,0.13}{##1}}}
\expandafter\def\csname PY@tok@dl\endcsname{\def\PY@tc##1{\textcolor[rgb]{0.73,0.13,0.13}{##1}}}
\expandafter\def\csname PY@tok@s2\endcsname{\def\PY@tc##1{\textcolor[rgb]{0.73,0.13,0.13}{##1}}}
\expandafter\def\csname PY@tok@sh\endcsname{\def\PY@tc##1{\textcolor[rgb]{0.73,0.13,0.13}{##1}}}
\expandafter\def\csname PY@tok@s1\endcsname{\def\PY@tc##1{\textcolor[rgb]{0.73,0.13,0.13}{##1}}}
\expandafter\def\csname PY@tok@mb\endcsname{\def\PY@tc##1{\textcolor[rgb]{0.40,0.40,0.40}{##1}}}
\expandafter\def\csname PY@tok@mf\endcsname{\def\PY@tc##1{\textcolor[rgb]{0.40,0.40,0.40}{##1}}}
\expandafter\def\csname PY@tok@mh\endcsname{\def\PY@tc##1{\textcolor[rgb]{0.40,0.40,0.40}{##1}}}
\expandafter\def\csname PY@tok@mi\endcsname{\def\PY@tc##1{\textcolor[rgb]{0.40,0.40,0.40}{##1}}}
\expandafter\def\csname PY@tok@il\endcsname{\def\PY@tc##1{\textcolor[rgb]{0.40,0.40,0.40}{##1}}}
\expandafter\def\csname PY@tok@mo\endcsname{\def\PY@tc##1{\textcolor[rgb]{0.40,0.40,0.40}{##1}}}
\expandafter\def\csname PY@tok@ch\endcsname{\let\PY@it=\textit\def\PY@tc##1{\textcolor[rgb]{0.25,0.50,0.50}{##1}}}
\expandafter\def\csname PY@tok@cm\endcsname{\let\PY@it=\textit\def\PY@tc##1{\textcolor[rgb]{0.25,0.50,0.50}{##1}}}
\expandafter\def\csname PY@tok@cpf\endcsname{\let\PY@it=\textit\def\PY@tc##1{\textcolor[rgb]{0.25,0.50,0.50}{##1}}}
\expandafter\def\csname PY@tok@c1\endcsname{\let\PY@it=\textit\def\PY@tc##1{\textcolor[rgb]{0.25,0.50,0.50}{##1}}}
\expandafter\def\csname PY@tok@cs\endcsname{\let\PY@it=\textit\def\PY@tc##1{\textcolor[rgb]{0.25,0.50,0.50}{##1}}}

\def\PYZbs{\char`\\}
\def\PYZus{\char`\_}
\def\PYZob{\char`\{}
\def\PYZcb{\char`\}}
\def\PYZca{\char`\^}
\def\PYZam{\char`\&}
\def\PYZlt{\char`\<}
\def\PYZgt{\char`\>}
\def\PYZsh{\char`\#}
\def\PYZpc{\char`\%}
\def\PYZdl{\char`\$}
\def\PYZhy{\char`\-}
\def\PYZsq{\char`\'}
\def\PYZdq{\char`\"}
\def\PYZti{\char`\~}
% for compatibility with earlier versions
\def\PYZat{@}
\def\PYZlb{[}
\def\PYZrb{]}
\makeatother


    % Exact colors from NB
    \definecolor{incolor}{rgb}{0.0, 0.0, 0.5}
    \definecolor{outcolor}{rgb}{0.545, 0.0, 0.0}



    
    % Prevent overflowing lines due to hard-to-break entities
    \sloppy 
    % Setup hyperref package
    \hypersetup{
      breaklinks=true,  % so long urls are correctly broken across lines
      colorlinks=true,
      urlcolor=urlcolor,
      linkcolor=linkcolor,
      citecolor=citecolor,
      }
    % Slightly bigger margins than the latex defaults
    
    \geometry{verbose,tmargin=1in,bmargin=1in,lmargin=1in,rmargin=1in}
    
    

    \begin{document}
    
    
    \maketitle
    
    

    
    \begin{verbatim}
<h1>TUGAS III - Individu</h1>
<h2>DATA WAREHOUSE</h2>
<h3>Maftuh Mashuri (11160940000076)</h3>
\end{verbatim}

    \section{Fungsi untuk koneksi
database}\label{fungsi-untuk-koneksi-database}

    \begin{Verbatim}[commandchars=\\\{\}]
{\color{incolor}In [{\color{incolor}183}]:} \PY{k}{def} \PY{n+nf}{connect}\PY{p}{(}\PY{n}{nama\PYZus{}db} \PY{o}{=} \PY{l+s+s2}{\PYZdq{}}\PY{l+s+s2}{kantor}\PY{l+s+s2}{\PYZdq{}}\PY{p}{,} \PY{n}{password} \PY{o}{=} \PY{l+s+s2}{\PYZdq{}}\PY{l+s+s2}{maftuh2003}\PY{l+s+s2}{\PYZdq{}}\PY{p}{)}\PY{p}{:}
              \PY{k+kn}{import} \PY{n+nn}{psycopg2}
              \PY{n}{conn} \PY{o}{=} \PY{n}{psycopg2}\PY{o}{.}\PY{n}{connect}\PY{p}{(}\PY{n}{database} \PY{o}{=} \PY{n}{nama\PYZus{}db}\PY{p}{,} \PY{n}{user} \PY{o}{=} \PY{l+s+s2}{\PYZdq{}}\PY{l+s+s2}{postgres}\PY{l+s+s2}{\PYZdq{}}\PY{p}{,} \PY{n}{password} \PY{o}{=} \PY{n}{password}\PY{p}{,} \PY{n}{host} \PY{o}{=} \PY{l+s+s2}{\PYZdq{}}\PY{l+s+s2}{localhost}\PY{l+s+s2}{\PYZdq{}}\PY{p}{,} \PY{n}{port} \PY{o}{=} \PY{l+s+s2}{\PYZdq{}}\PY{l+s+s2}{5432}\PY{l+s+s2}{\PYZdq{}}\PY{p}{)}
              \PY{k}{return} \PY{n}{conn}
\end{Verbatim}


    \section{Fungsi untuk CRUD}\label{fungsi-untuk-crud}

Fungsi ini untuk melakukan running query ke database dengan input query
yaitu string berisi query perintah untuk database dan select dengan type
data boolean karena hanya perintah SELECT yang mengeluarkan output

    \begin{Verbatim}[commandchars=\\\{\}]
{\color{incolor}In [{\color{incolor}184}]:} \PY{k}{def} \PY{n+nf}{execute}\PY{p}{(}\PY{n}{query}\PY{p}{,} \PY{n}{select} \PY{o}{=} \PY{k+kc}{True}\PY{p}{)}\PY{p}{:}
              \PY{n}{conn} \PY{o}{=} \PY{n}{connect}\PY{p}{(}\PY{p}{)}
              \PY{n}{cur} \PY{o}{=} \PY{n}{conn}\PY{o}{.}\PY{n}{cursor}\PY{p}{(}\PY{p}{)}
              \PY{n}{cur}\PY{o}{.}\PY{n}{execute}\PY{p}{(}\PY{n}{query}\PY{p}{)}
              \PY{k}{if} \PY{n}{select}\PY{p}{:}
                  \PY{k}{return} \PY{n}{cur}\PY{o}{.}\PY{n}{fetchall}\PY{p}{(}\PY{p}{)}
              \PY{k}{else}\PY{p}{:}
                  \PY{n}{conn}\PY{o}{.}\PY{n}{commit}\PY{p}{(}\PY{p}{)}
              \PY{n}{conn}\PY{o}{.}\PY{n}{close}\PY{p}{(}\PY{p}{)}
\end{Verbatim}


    \begin{Verbatim}[commandchars=\\\{\}]
{\color{incolor}In [{\color{incolor}201}]:} \PY{c+c1}{\PYZsh{} query untuk mengosongkan tabel}
          \PY{n}{query} \PY{o}{=} \PY{l+s+s2}{\PYZdq{}}\PY{l+s+s2}{DELETE FROM karyawan; DELETE FROM cuti\PYZus{}karyawan;}\PY{l+s+s2}{\PYZdq{}}
          \PY{n}{execute}\PY{p}{(}\PY{n}{query}\PY{p}{,} \PY{k+kc}{False}\PY{p}{)}
\end{Verbatim}


    \begin{Verbatim}[commandchars=\\\{\}]
{\color{incolor}In [{\color{incolor}197}]:} \PY{c+c1}{\PYZsh{} query untuk mengosongkan tabel}
          \PY{n}{query} \PY{o}{=} \PY{l+s+s2}{\PYZdq{}}\PY{l+s+s2}{DROP TABLE karyawan; DROP TABLE cuti\PYZus{}karyawan;}\PY{l+s+s2}{\PYZdq{}}
          \PY{n}{execute}\PY{p}{(}\PY{n}{query}\PY{p}{,} \PY{k+kc}{False}\PY{p}{)}
\end{Verbatim}


    \section{b) Membuat tabel-tabel pada database
kantor.}\label{b-membuat-tabel-tabel-pada-database-kantor.}

    \begin{Verbatim}[commandchars=\\\{\}]
{\color{incolor}In [{\color{incolor}200}]:} \PY{n}{query} \PY{o}{=} \PY{l+s+s1}{\PYZsq{}\PYZsq{}\PYZsq{}}
          \PY{l+s+s1}{    CREATE TABLE IF NOT EXISTS karyawan(}
          \PY{l+s+s1}{    Nomor\PYZus{}induk VARCHAR(10) NOT NULL,}
          \PY{l+s+s1}{    Nama VARCHAR(30),}
          \PY{l+s+s1}{    Alamat TEXT,}
          \PY{l+s+s1}{    Tanggal\PYZus{}lahir DATE,}
          \PY{l+s+s1}{    Tanggal\PYZus{}masuk DATE,}
          \PY{l+s+s1}{    PRIMARY KEY(Nomor\PYZus{}induk)}
          \PY{l+s+s1}{    );}
          
          \PY{l+s+s1}{    CREATE TABLE IF NOT EXISTS cuti\PYZus{}karyawan(}
          \PY{l+s+s1}{    id INT NOT NULL,}
          \PY{l+s+s1}{    Nomor\PYZus{}induk VARCHAR(10),}
          \PY{l+s+s1}{    Tanggal\PYZus{}mulai DATE,}
          \PY{l+s+s1}{    Lama\PYZus{}cuti SMALLINT,}
          \PY{l+s+s1}{    Keterangan TEXT,}
          \PY{l+s+s1}{    PRIMARY KEY(id));}
          \PY{l+s+s1}{    }\PY{l+s+s1}{\PYZsq{}\PYZsq{}\PYZsq{}}
          \PY{n}{execute}\PY{p}{(}\PY{n}{query}\PY{p}{,} \PY{k+kc}{False}\PY{p}{)}
          \PY{n+nb}{print}\PY{p}{(}\PY{l+s+s2}{\PYZdq{}}\PY{l+s+s2}{Membuat tabel berhasil}\PY{l+s+s2}{\PYZdq{}}\PY{p}{)}
\end{Verbatim}


    \begin{Verbatim}[commandchars=\\\{\}]
Membuat tabel berhasil

    \end{Verbatim}

    \section{c) Mengisi tabel-tabel}\label{c-mengisi-tabel-tabel}

    \begin{Verbatim}[commandchars=\\\{\}]
{\color{incolor}In [{\color{incolor}202}]:} \PY{n}{query} \PY{o}{=} \PY{l+s+s1}{\PYZsq{}\PYZsq{}\PYZsq{}}
          \PY{l+s+s1}{    INSERT INTO karyawan (Nomor\PYZus{}induk, Nama, Alamat, Tanggal\PYZus{}lahir, Tanggal\PYZus{}masuk)}
          \PY{l+s+s1}{    VALUES }
          \PY{l+s+s1}{    (}\PY{l+s+s1}{\PYZsq{}}\PY{l+s+s1}{IP06001}\PY{l+s+s1}{\PYZsq{}}\PY{l+s+s1}{, }\PY{l+s+s1}{\PYZsq{}}\PY{l+s+s1}{Agus}\PY{l+s+s1}{\PYZsq{}}\PY{l+s+s1}{, }\PY{l+s+s1}{\PYZsq{}}\PY{l+s+s1}{Jln. Gajah Mada 115A, Jakarta Pusat}\PY{l+s+s1}{\PYZsq{}}\PY{l+s+s1}{, }\PY{l+s+s1}{\PYZsq{}}\PY{l+s+s1}{1970\PYZhy{}08\PYZhy{}01}\PY{l+s+s1}{\PYZsq{}}\PY{l+s+s1}{, }\PY{l+s+s1}{\PYZsq{}}\PY{l+s+s1}{2006\PYZhy{}07\PYZhy{}07}\PY{l+s+s1}{\PYZsq{}}\PY{l+s+s1}{),}
          \PY{l+s+s1}{    (}\PY{l+s+s1}{\PYZsq{}}\PY{l+s+s1}{IP06002}\PY{l+s+s1}{\PYZsq{}}\PY{l+s+s1}{, }\PY{l+s+s1}{\PYZsq{}}\PY{l+s+s1}{Amin}\PY{l+s+s1}{\PYZsq{}}\PY{l+s+s1}{, }\PY{l+s+s1}{\PYZsq{}}\PY{l+s+s1}{Jln. Bungur sari v No. 178 Bandung}\PY{l+s+s1}{\PYZsq{}}\PY{l+s+s1}{, }\PY{l+s+s1}{\PYZsq{}}\PY{l+s+s1}{1977\PYZhy{}05\PYZhy{}03}\PY{l+s+s1}{\PYZsq{}}\PY{l+s+s1}{, }\PY{l+s+s1}{\PYZsq{}}\PY{l+s+s1}{2006\PYZhy{}07\PYZhy{}07}\PY{l+s+s1}{\PYZsq{}}\PY{l+s+s1}{),}
          \PY{l+s+s1}{    (}\PY{l+s+s1}{\PYZsq{}}\PY{l+s+s1}{IP06003}\PY{l+s+s1}{\PYZsq{}}\PY{l+s+s1}{, }\PY{l+s+s1}{\PYZsq{}}\PY{l+s+s1}{Yusuf}\PY{l+s+s1}{\PYZsq{}}\PY{l+s+s1}{, }\PY{l+s+s1}{\PYZsq{}}\PY{l+s+s1}{Jln. Yosodpuro 15, Surabaya}\PY{l+s+s1}{\PYZsq{}}\PY{l+s+s1}{, }\PY{l+s+s1}{\PYZsq{}}\PY{l+s+s1}{1973\PYZhy{}08\PYZhy{}09}\PY{l+s+s1}{\PYZsq{}}\PY{l+s+s1}{, }\PY{l+s+s1}{\PYZsq{}}\PY{l+s+s1}{2006\PYZhy{}07\PYZhy{}07}\PY{l+s+s1}{\PYZsq{}}\PY{l+s+s1}{),}
          \PY{l+s+s1}{    (}\PY{l+s+s1}{\PYZsq{}}\PY{l+s+s1}{IP07004}\PY{l+s+s1}{\PYZsq{}}\PY{l+s+s1}{, }\PY{l+s+s1}{\PYZsq{}}\PY{l+s+s1}{Alyssa}\PY{l+s+s1}{\PYZsq{}}\PY{l+s+s1}{, }\PY{l+s+s1}{\PYZsq{}}\PY{l+s+s1}{Jln. Cendana No. 6 Bandung}\PY{l+s+s1}{\PYZsq{}}\PY{l+s+s1}{, }\PY{l+s+s1}{\PYZsq{}}\PY{l+s+s1}{1983\PYZhy{}02\PYZhy{}14}\PY{l+s+s1}{\PYZsq{}}\PY{l+s+s1}{, }\PY{l+s+s1}{\PYZsq{}}\PY{l+s+s1}{2007\PYZhy{}01\PYZhy{}05}\PY{l+s+s1}{\PYZsq{}}\PY{l+s+s1}{),}
          \PY{l+s+s1}{    (}\PY{l+s+s1}{\PYZsq{}}\PY{l+s+s1}{IP07005}\PY{l+s+s1}{\PYZsq{}}\PY{l+s+s1}{, }\PY{l+s+s1}{\PYZsq{}}\PY{l+s+s1}{Maulana}\PY{l+s+s1}{\PYZsq{}}\PY{l+s+s1}{, }\PY{l+s+s1}{\PYZsq{}}\PY{l+s+s1}{Jln. Ampera Raya No 1}\PY{l+s+s1}{\PYZsq{}}\PY{l+s+s1}{, }\PY{l+s+s1}{\PYZsq{}}\PY{l+s+s1}{1985\PYZhy{}10\PYZhy{}10}\PY{l+s+s1}{\PYZsq{}}\PY{l+s+s1}{, }\PY{l+s+s1}{\PYZsq{}}\PY{l+s+s1}{2007\PYZhy{}02\PYZhy{}05}\PY{l+s+s1}{\PYZsq{}}\PY{l+s+s1}{),}
          \PY{l+s+s1}{    (}\PY{l+s+s1}{\PYZsq{}}\PY{l+s+s1}{IP07006}\PY{l+s+s1}{\PYZsq{}}\PY{l+s+s1}{, }\PY{l+s+s1}{\PYZsq{}}\PY{l+s+s1}{Afika}\PY{l+s+s1}{\PYZsq{}}\PY{l+s+s1}{, }\PY{l+s+s1}{\PYZsq{}}\PY{l+s+s1}{Jln. Pejaten Barat No 6A}\PY{l+s+s1}{\PYZsq{}}\PY{l+s+s1}{, }\PY{l+s+s1}{\PYZsq{}}\PY{l+s+s1}{1987\PYZhy{}03\PYZhy{}09}\PY{l+s+s1}{\PYZsq{}}\PY{l+s+s1}{, }\PY{l+s+s1}{\PYZsq{}}\PY{l+s+s1}{2007\PYZhy{}06\PYZhy{}09}\PY{l+s+s1}{\PYZsq{}}\PY{l+s+s1}{),}
          \PY{l+s+s1}{    (}\PY{l+s+s1}{\PYZsq{}}\PY{l+s+s1}{IP07007}\PY{l+s+s1}{\PYZsq{}}\PY{l+s+s1}{, }\PY{l+s+s1}{\PYZsq{}}\PY{l+s+s1}{James}\PY{l+s+s1}{\PYZsq{}}\PY{l+s+s1}{, }\PY{l+s+s1}{\PYZsq{}}\PY{l+s+s1}{Jln. Padjadjaran No. 111, Bandung}\PY{l+s+s1}{\PYZsq{}}\PY{l+s+s1}{, }\PY{l+s+s1}{\PYZsq{}}\PY{l+s+s1}{1988\PYZhy{}05\PYZhy{}19}\PY{l+s+s1}{\PYZsq{}}\PY{l+s+s1}{, }\PY{l+s+s1}{\PYZsq{}}\PY{l+s+s1}{2007\PYZhy{}06\PYZhy{}09}\PY{l+s+s1}{\PYZsq{}}\PY{l+s+s1}{),}
          \PY{l+s+s1}{    (}\PY{l+s+s1}{\PYZsq{}}\PY{l+s+s1}{IP09008}\PY{l+s+s1}{\PYZsq{}}\PY{l+s+s1}{, }\PY{l+s+s1}{\PYZsq{}}\PY{l+s+s1}{Octavanus}\PY{l+s+s1}{\PYZsq{}}\PY{l+s+s1}{, }\PY{l+s+s1}{\PYZsq{}}\PY{l+s+s1}{Jln. Gajah Mada 101, Semarang}\PY{l+s+s1}{\PYZsq{}}\PY{l+s+s1}{, }\PY{l+s+s1}{\PYZsq{}}\PY{l+s+s1}{1988\PYZhy{}10\PYZhy{}07}\PY{l+s+s1}{\PYZsq{}}\PY{l+s+s1}{, }\PY{l+s+s1}{\PYZsq{}}\PY{l+s+s1}{2008\PYZhy{}08\PYZhy{}08}\PY{l+s+s1}{\PYZsq{}}\PY{l+s+s1}{),}
          \PY{l+s+s1}{    (}\PY{l+s+s1}{\PYZsq{}}\PY{l+s+s1}{IP09009}\PY{l+s+s1}{\PYZsq{}}\PY{l+s+s1}{, }\PY{l+s+s1}{\PYZsq{}}\PY{l+s+s1}{Nugroho}\PY{l+s+s1}{\PYZsq{}}\PY{l+s+s1}{, }\PY{l+s+s1}{\PYZsq{}}\PY{l+s+s1}{Jln. Duren Tiga 196, Jakarta Selatan}\PY{l+s+s1}{\PYZsq{}}\PY{l+s+s1}{, }\PY{l+s+s1}{\PYZsq{}}\PY{l+s+s1}{1988\PYZhy{}01\PYZhy{}20}\PY{l+s+s1}{\PYZsq{}}\PY{l+s+s1}{, }\PY{l+s+s1}{\PYZsq{}}\PY{l+s+s1}{2008\PYZhy{}11\PYZhy{}11}\PY{l+s+s1}{\PYZsq{}}\PY{l+s+s1}{),}
          \PY{l+s+s1}{    (}\PY{l+s+s1}{\PYZsq{}}\PY{l+s+s1}{IP090010}\PY{l+s+s1}{\PYZsq{}}\PY{l+s+s1}{, }\PY{l+s+s1}{\PYZsq{}}\PY{l+s+s1}{Raisa}\PY{l+s+s1}{\PYZsq{}}\PY{l+s+s1}{, }\PY{l+s+s1}{\PYZsq{}}\PY{l+s+s1}{Jln. Nangka Jakarta Selatan}\PY{l+s+s1}{\PYZsq{}}\PY{l+s+s1}{, }\PY{l+s+s1}{\PYZsq{}}\PY{l+s+s1}{1989\PYZhy{}12\PYZhy{}29}\PY{l+s+s1}{\PYZsq{}}\PY{l+s+s1}{, }\PY{l+s+s1}{\PYZsq{}}\PY{l+s+s1}{2009\PYZhy{}02\PYZhy{}09}\PY{l+s+s1}{\PYZsq{}}\PY{l+s+s1}{);}
          \PY{l+s+s1}{    }
          \PY{l+s+s1}{    INSERT INTO cuti\PYZus{}karyawan(id, Nomor\PYZus{}induk, Tanggal\PYZus{}mulai, Lama\PYZus{}cuti, Keterangan)}
          \PY{l+s+s1}{    VALUES}
          \PY{l+s+s1}{    (1, }\PY{l+s+s1}{\PYZsq{}}\PY{l+s+s1}{IP06001}\PY{l+s+s1}{\PYZsq{}}\PY{l+s+s1}{, }\PY{l+s+s1}{\PYZsq{}}\PY{l+s+s1}{2012\PYZhy{}02\PYZhy{}01}\PY{l+s+s1}{\PYZsq{}}\PY{l+s+s1}{, 3, }\PY{l+s+s1}{\PYZsq{}}\PY{l+s+s1}{Acara keluar}\PY{l+s+s1}{\PYZsq{}}\PY{l+s+s1}{),}
          \PY{l+s+s1}{    (2, }\PY{l+s+s1}{\PYZsq{}}\PY{l+s+s1}{IP06001}\PY{l+s+s1}{\PYZsq{}}\PY{l+s+s1}{, }\PY{l+s+s1}{\PYZsq{}}\PY{l+s+s1}{2012\PYZhy{}02\PYZhy{}13}\PY{l+s+s1}{\PYZsq{}}\PY{l+s+s1}{, 2, }\PY{l+s+s1}{\PYZsq{}}\PY{l+s+s1}{Anak sakit}\PY{l+s+s1}{\PYZsq{}}\PY{l+s+s1}{),}
          \PY{l+s+s1}{    (3, }\PY{l+s+s1}{\PYZsq{}}\PY{l+s+s1}{IP07007}\PY{l+s+s1}{\PYZsq{}}\PY{l+s+s1}{, }\PY{l+s+s1}{\PYZsq{}}\PY{l+s+s1}{2012\PYZhy{}02\PYZhy{}15}\PY{l+s+s1}{\PYZsq{}}\PY{l+s+s1}{, 1, }\PY{l+s+s1}{\PYZsq{}}\PY{l+s+s1}{Nenek sakit}\PY{l+s+s1}{\PYZsq{}}\PY{l+s+s1}{),}
          \PY{l+s+s1}{    (4, }\PY{l+s+s1}{\PYZsq{}}\PY{l+s+s1}{IP06003}\PY{l+s+s1}{\PYZsq{}}\PY{l+s+s1}{, }\PY{l+s+s1}{\PYZsq{}}\PY{l+s+s1}{2012\PYZhy{}02\PYZhy{}17}\PY{l+s+s1}{\PYZsq{}}\PY{l+s+s1}{, 1, }\PY{l+s+s1}{\PYZsq{}}\PY{l+s+s1}{Mendaftar sekolah anak}\PY{l+s+s1}{\PYZsq{}}\PY{l+s+s1}{),}
          \PY{l+s+s1}{    (5, }\PY{l+s+s1}{\PYZsq{}}\PY{l+s+s1}{IP07006}\PY{l+s+s1}{\PYZsq{}}\PY{l+s+s1}{, }\PY{l+s+s1}{\PYZsq{}}\PY{l+s+s1}{2012\PYZhy{}02\PYZhy{}20}\PY{l+s+s1}{\PYZsq{}}\PY{l+s+s1}{, 5, }\PY{l+s+s1}{\PYZsq{}}\PY{l+s+s1}{Menikah}\PY{l+s+s1}{\PYZsq{}}\PY{l+s+s1}{),}
          \PY{l+s+s1}{    (6, }\PY{l+s+s1}{\PYZsq{}}\PY{l+s+s1}{IP07004}\PY{l+s+s1}{\PYZsq{}}\PY{l+s+s1}{, }\PY{l+s+s1}{\PYZsq{}}\PY{l+s+s1}{2012\PYZhy{}02\PYZhy{}27}\PY{l+s+s1}{\PYZsq{}}\PY{l+s+s1}{, 1, }\PY{l+s+s1}{\PYZsq{}}\PY{l+s+s1}{Imunisasi anak}\PY{l+s+s1}{\PYZsq{}}\PY{l+s+s1}{);}
          \PY{l+s+s1}{    }\PY{l+s+s1}{\PYZsq{}\PYZsq{}\PYZsq{}}
          \PY{n}{execute}\PY{p}{(}\PY{n}{query}\PY{p}{,} \PY{k+kc}{False}\PY{p}{)}
          \PY{n+nb}{print}\PY{p}{(}\PY{l+s+s2}{\PYZdq{}}\PY{l+s+s2}{Input data berhasil}\PY{l+s+s2}{\PYZdq{}}\PY{p}{)}
\end{Verbatim}


    \begin{Verbatim}[commandchars=\\\{\}]
Input data berhasil

    \end{Verbatim}

    \section{d) Tentukan empat orang karyawan yang pertama kali masuk
(bergabung)}\label{d-tentukan-empat-orang-karyawan-yang-pertama-kali-masuk-bergabung}

    \begin{Verbatim}[commandchars=\\\{\}]
{\color{incolor}In [{\color{incolor}203}]:} \PY{n}{query} \PY{o}{=} \PY{l+s+s2}{\PYZdq{}}\PY{l+s+s2}{SELECT Nama FROM karyawan ORDER BY Tanggal\PYZus{}masuk LIMIT 4;}\PY{l+s+s2}{\PYZdq{}}
          \PY{n}{data} \PY{o}{=} \PY{n}{execute}\PY{p}{(}\PY{n}{query}\PY{p}{)}
          
          \PY{n+nb}{print}\PY{p}{(}\PY{l+s+s2}{\PYZdq{}}\PY{l+s+s2}{Empat orang karyawan yang pertama kali masuk adalah : }\PY{l+s+s2}{\PYZdq{}}\PY{p}{)}
          \PY{k}{for} \PY{n}{user} \PY{o+ow}{in} \PY{n}{data}\PY{p}{:}
              \PY{n+nb}{print}\PY{p}{(}\PY{l+s+s2}{\PYZdq{}}\PY{l+s+s2}{\PYZhy{}}\PY{l+s+s2}{\PYZdq{}}\PY{p}{,} \PY{n}{user}\PY{p}{[}\PY{l+m+mi}{0}\PY{p}{]}\PY{p}{)}
\end{Verbatim}


    \begin{Verbatim}[commandchars=\\\{\}]
Empat orang karyawan yang pertama kali masuk adalah : 
- Agus
- Yusuf
- Amin
- Alyssa

    \end{Verbatim}

    \section{e) Tentukan dua orang karyawan yang terakhir kali masuk
(bergabung)}\label{e-tentukan-dua-orang-karyawan-yang-terakhir-kali-masuk-bergabung}

    \begin{Verbatim}[commandchars=\\\{\}]
{\color{incolor}In [{\color{incolor}204}]:} \PY{n}{query} \PY{o}{=} \PY{l+s+s2}{\PYZdq{}}\PY{l+s+s2}{SELECT Nama FROM karyawan ORDER BY Tanggal\PYZus{}masuk DESC LIMIT 2;}\PY{l+s+s2}{\PYZdq{}}
          \PY{n}{data} \PY{o}{=} \PY{n}{execute}\PY{p}{(}\PY{n}{query}\PY{p}{)}
          
          \PY{n+nb}{print}\PY{p}{(}\PY{l+s+s2}{\PYZdq{}}\PY{l+s+s2}{Dua orang karyawan yang terakhir kali masuk adalah : }\PY{l+s+s2}{\PYZdq{}}\PY{p}{)}
          \PY{k}{for} \PY{n}{user} \PY{o+ow}{in} \PY{n}{data}\PY{p}{:}
              \PY{n+nb}{print}\PY{p}{(}\PY{l+s+s2}{\PYZdq{}}\PY{l+s+s2}{\PYZhy{}}\PY{l+s+s2}{\PYZdq{}}\PY{p}{,} \PY{n}{user}\PY{p}{[}\PY{l+m+mi}{0}\PY{p}{]}\PY{p}{)}
\end{Verbatim}


    \begin{Verbatim}[commandchars=\\\{\}]
Dua orang karyawan yang terakhir kali masuk adalah : 
- Raisa
- Nugroho

    \end{Verbatim}

    \section{f) Tentukan nama karyawan yang paling banyak mengambil cuti
beserta keterangan
cutinya.}\label{f-tentukan-nama-karyawan-yang-paling-banyak-mengambil-cuti-beserta-keterangan-cutinya.}

    \begin{Verbatim}[commandchars=\\\{\}]
{\color{incolor}In [{\color{incolor}205}]:} \PY{n}{query} \PY{o}{=} \PY{l+s+s1}{\PYZsq{}\PYZsq{}\PYZsq{}}
          \PY{l+s+s1}{    SELECT }
          \PY{l+s+s1}{        karyawan.Nama, cuti\PYZus{}karyawan.keterangan}
          \PY{l+s+s1}{    FROM }
          \PY{l+s+s1}{        karyawan}
          \PY{l+s+s1}{    JOIN }
          \PY{l+s+s1}{        cuti\PYZus{}karyawan }
          \PY{l+s+s1}{    ON }
          \PY{l+s+s1}{        cuti\PYZus{}karyawan.Nomor\PYZus{}induk = karyawan.Nomor\PYZus{}induk}
          \PY{l+s+s1}{    JOIN }
          \PY{l+s+s1}{        (SELECT Nomor\PYZus{}induk, COUNT(Nomor\PYZus{}induk) AS total\PYZus{}cuti FROM cuti\PYZus{}karyawan GROUP BY Nomor\PYZus{}induk) grup\PYZus{}cuti}
          \PY{l+s+s1}{    ON }
          \PY{l+s+s1}{        grup\PYZus{}cuti.Nomor\PYZus{}induk = karyawan.Nomor\PYZus{}induk}
          \PY{l+s+s1}{    WHERE }
          \PY{l+s+s1}{        grup\PYZus{}cuti.total\PYZus{}cuti = }
          \PY{l+s+s1}{            (SELECT MAX(x.max) FROM (SELECT COUNT(Nomor\PYZus{}induk) AS max FROM cuti\PYZus{}karyawan GROUP BY Nomor\PYZus{}induk) x);}
          \PY{l+s+s1}{    }\PY{l+s+s1}{\PYZsq{}\PYZsq{}\PYZsq{}}
          \PY{n}{data} \PY{o}{=} \PY{n}{execute}\PY{p}{(}\PY{n}{query}\PY{p}{)}
          
          \PY{n+nb}{print}\PY{p}{(}\PY{l+s+s2}{\PYZdq{}}\PY{l+s+s2}{Karyawan yang paling banyak mengambil cuti adalah:}\PY{l+s+s2}{\PYZdq{}}\PY{p}{)}
          \PY{k}{for} \PY{n}{i} \PY{o+ow}{in} \PY{n+nb}{range}\PY{p}{(}\PY{n+nb}{len}\PY{p}{(}\PY{n}{data}\PY{p}{)}\PY{p}{)}\PY{p}{:}
              \PY{k}{if} \PY{n}{i} \PY{o}{\PYZlt{}} \PY{n+nb}{len}\PY{p}{(}\PY{n}{data}\PY{p}{)}\PY{o}{\PYZhy{}}\PY{l+m+mi}{1}\PY{p}{:}
                  \PY{k}{if} \PY{n}{data}\PY{p}{[}\PY{n}{i}\PY{p}{]}\PY{p}{[}\PY{l+m+mi}{0}\PY{p}{]} \PY{o}{==} \PY{n}{data}\PY{p}{[}\PY{n}{i}\PY{o}{+}\PY{l+m+mi}{1}\PY{p}{]}\PY{p}{[}\PY{l+m+mi}{0}\PY{p}{]}\PY{p}{:}
                      \PY{n+nb}{print}\PY{p}{(}\PY{l+s+s2}{\PYZdq{}}\PY{l+s+s2}{\PYZhy{}\PYZgt{} }\PY{l+s+s2}{\PYZdq{}} \PY{o}{+} \PY{n}{data}\PY{p}{[}\PY{n}{i}\PY{p}{]}\PY{p}{[}\PY{l+m+mi}{0}\PY{p}{]} \PY{o}{+} \PY{l+s+s2}{\PYZdq{}}\PY{l+s+s2}{, dengan keterangan:}\PY{l+s+s2}{\PYZdq{}}\PY{p}{)}
              \PY{n+nb}{print}\PY{p}{(}\PY{l+s+s2}{\PYZdq{}}\PY{l+s+s2}{ \PYZhy{}}\PY{l+s+s2}{\PYZdq{}}\PY{p}{,} \PY{n}{data}\PY{p}{[}\PY{n}{i}\PY{p}{]}\PY{p}{[}\PY{l+m+mi}{1}\PY{p}{]}\PY{p}{)}
\end{Verbatim}


    \begin{Verbatim}[commandchars=\\\{\}]
Karyawan yang paling banyak mengambil cuti adalah:
-> Agus, dengan keterangan:
 - Acara keluar
 - Anak sakit

    \end{Verbatim}

    \section{g) Tentukan nama karayawan yang memiliki umur paling
tua}\label{g-tentukan-nama-karayawan-yang-memiliki-umur-paling-tua}

    \begin{Verbatim}[commandchars=\\\{\}]
{\color{incolor}In [{\color{incolor}206}]:} \PY{n}{query} \PY{o}{=} \PY{l+s+s2}{\PYZdq{}}\PY{l+s+s2}{SELECT Nama FROM karyawan WHERE Tanggal\PYZus{}lahir = (SELECT MIN(Tanggal\PYZus{}lahir) FROM karyawan);}\PY{l+s+s2}{\PYZdq{}}
          \PY{n}{data} \PY{o}{=} \PY{n}{execute}\PY{p}{(}\PY{n}{query}\PY{p}{)}
          
          \PY{n+nb}{print}\PY{p}{(}\PY{l+s+s2}{\PYZdq{}}\PY{l+s+s2}{Karyawan yang memiliki umur paling tua adalah:}\PY{l+s+s2}{\PYZdq{}}\PY{p}{)}
          \PY{k}{for} \PY{n}{user} \PY{o+ow}{in} \PY{n}{data}\PY{p}{:}
              \PY{n+nb}{print}\PY{p}{(}\PY{l+s+s2}{\PYZdq{}}\PY{l+s+s2}{\PYZhy{}}\PY{l+s+s2}{\PYZdq{}}\PY{p}{,} \PY{n}{user}\PY{p}{[}\PY{l+m+mi}{0}\PY{p}{]}\PY{p}{)}
\end{Verbatim}


    \begin{Verbatim}[commandchars=\\\{\}]
Karyawan yang memiliki umur paling tua adalah:
- Agus

    \end{Verbatim}

    \section{h) Tentukan nama karyawan yang memiliki umur paling
muda}\label{h-tentukan-nama-karyawan-yang-memiliki-umur-paling-muda}

    \begin{Verbatim}[commandchars=\\\{\}]
{\color{incolor}In [{\color{incolor}207}]:} \PY{n}{query} \PY{o}{=} \PY{l+s+s2}{\PYZdq{}}\PY{l+s+s2}{SELECT Nama FROM karyawan WHERE Tanggal\PYZus{}lahir = (SELECT MAX(Tanggal\PYZus{}lahir) FROM karyawan);}\PY{l+s+s2}{\PYZdq{}}
          \PY{n}{data} \PY{o}{=} \PY{n}{execute}\PY{p}{(}\PY{n}{query}\PY{p}{)}
          
          \PY{n+nb}{print}\PY{p}{(}\PY{l+s+s2}{\PYZdq{}}\PY{l+s+s2}{Karyawan yang memiliki umur paling muda adalah:}\PY{l+s+s2}{\PYZdq{}}\PY{p}{)}
          \PY{k}{for} \PY{n}{user} \PY{o+ow}{in} \PY{n}{data}\PY{p}{:}
              \PY{n+nb}{print}\PY{p}{(}\PY{l+s+s2}{\PYZdq{}}\PY{l+s+s2}{\PYZhy{}}\PY{l+s+s2}{\PYZdq{}}\PY{p}{,} \PY{n}{user}\PY{p}{[}\PY{l+m+mi}{0}\PY{p}{]}\PY{p}{)}
\end{Verbatim}


    \begin{Verbatim}[commandchars=\\\{\}]
Karyawan yang memiliki umur paling muda adalah:
- Raisa

    \end{Verbatim}

    \section{i) Tentukan nama-nama karyawan yang paling dulu masuk kerja
setelah
cuti}\label{i-tentukan-nama-nama-karyawan-yang-paling-dulu-masuk-kerja-setelah-cuti}

    \begin{Verbatim}[commandchars=\\\{\}]
{\color{incolor}In [{\color{incolor}208}]:} \PY{n}{query} \PY{o}{=} \PY{l+s+s1}{\PYZsq{}\PYZsq{}\PYZsq{}}
          \PY{l+s+s1}{    SELECT Nama}
          \PY{l+s+s1}{    FROM karyawan}
          \PY{l+s+s1}{    JOIN cuti\PYZus{}karyawan ON cuti\PYZus{}karyawan.Nomor\PYZus{}induk = karyawan.Nomor\PYZus{}induk}
          \PY{l+s+s1}{    WHERE cuti\PYZus{}karyawan.Lama\PYZus{}cuti = (SELECT MIN(Lama\PYZus{}cuti) FROM cuti\PYZus{}karyawan);}
          \PY{l+s+s1}{    }\PY{l+s+s1}{\PYZsq{}\PYZsq{}\PYZsq{}}
          \PY{n}{data} \PY{o}{=} \PY{n}{execute}\PY{p}{(}\PY{n}{query}\PY{p}{)}
          
          \PY{n+nb}{print}\PY{p}{(}\PY{l+s+s2}{\PYZdq{}}\PY{l+s+s2}{Karyawan yang paling dulu masuk kerja setelah cuti adalah:}\PY{l+s+s2}{\PYZdq{}}\PY{p}{)}
          \PY{k}{for} \PY{n}{user} \PY{o+ow}{in} \PY{n}{data}\PY{p}{:}
              \PY{n+nb}{print}\PY{p}{(}\PY{l+s+s2}{\PYZdq{}}\PY{l+s+s2}{\PYZhy{}}\PY{l+s+s2}{\PYZdq{}}\PY{p}{,} \PY{n}{user}\PY{p}{[}\PY{l+m+mi}{0}\PY{p}{]}\PY{p}{)}
\end{Verbatim}


    \begin{Verbatim}[commandchars=\\\{\}]
Karyawan yang paling dulu masuk kerja setelah cuti adalah:
- Yusuf
- Alyssa
- James

    \end{Verbatim}

    \section{j) Tentukan total cuti yang diambil para
karyawan.}\label{j-tentukan-total-cuti-yang-diambil-para-karyawan.}

    \begin{Verbatim}[commandchars=\\\{\}]
{\color{incolor}In [{\color{incolor}209}]:} \PY{n}{query} \PY{o}{=} \PY{l+s+s1}{\PYZsq{}\PYZsq{}\PYZsq{}}
          \PY{l+s+s1}{    SELECT karyawan.Nama, COUNT(cuti\PYZus{}karyawan.Nomor\PYZus{}induk)}
          \PY{l+s+s1}{    FROM karyawan}
          \PY{l+s+s1}{    JOIN cuti\PYZus{}karyawan ON cuti\PYZus{}karyawan.Nomor\PYZus{}induk = karyawan.Nomor\PYZus{}induk}
          \PY{l+s+s1}{    GROUP BY karyawan.Nomor\PYZus{}induk;}
          \PY{l+s+s1}{    }\PY{l+s+s1}{\PYZsq{}\PYZsq{}\PYZsq{}}
          \PY{n}{data} \PY{o}{=} \PY{n}{execute}\PY{p}{(}\PY{n}{query}\PY{p}{)}
          \PY{n+nb}{print}\PY{p}{(}\PY{l+s+s2}{\PYZdq{}}\PY{l+s+s2}{Total cuti yang diambil para karyawan adalah:}\PY{l+s+s2}{\PYZdq{}}\PY{p}{)}
          \PY{k}{for} \PY{n}{user} \PY{o+ow}{in} \PY{n}{data}\PY{p}{:}
              \PY{n+nb}{print}\PY{p}{(}\PY{l+s+s2}{\PYZdq{}}\PY{l+s+s2}{\PYZhy{}}\PY{l+s+s2}{\PYZdq{}}\PY{p}{,} \PY{n}{user}\PY{p}{[}\PY{l+m+mi}{0}\PY{p}{]}\PY{p}{,} \PY{l+s+s2}{\PYZdq{}}\PY{l+s+s2}{sebanyak}\PY{l+s+s2}{\PYZdq{}}\PY{p}{,} \PY{n}{user}\PY{p}{[}\PY{l+m+mi}{1}\PY{p}{]}\PY{p}{,} \PY{l+s+s2}{\PYZdq{}}\PY{l+s+s2}{kali}\PY{l+s+s2}{\PYZdq{}}\PY{p}{)}
          
          \PY{n+nb}{print}\PY{p}{(}\PY{l+s+s2}{\PYZdq{}}\PY{l+s+se}{\PYZbs{}n}\PY{l+s+s2}{Sisa karyawan lainnya tidak mengambil cuti.}\PY{l+s+s2}{\PYZdq{}}\PY{p}{)}
\end{Verbatim}


    \begin{Verbatim}[commandchars=\\\{\}]
Total cuti yang diambil para karyawan adalah:
- Yusuf sebanyak 1 kali
- Afika sebanyak 1 kali
- Alyssa sebanyak 1 kali
- Agus sebanyak 2 kali
- James sebanyak 1 kali

Sisa karyawan lainnya tidak mengambil cuti.

    \end{Verbatim}

    \section{k) Apakah ada karyawan yang tidak mengambil cuti? Jika iya
tentukan nama-nama karyawan
tersebut.}\label{k-apakah-ada-karyawan-yang-tidak-mengambil-cuti-jika-iya-tentukan-nama-nama-karyawan-tersebut.}

    \begin{Verbatim}[commandchars=\\\{\}]
{\color{incolor}In [{\color{incolor}210}]:} \PY{n}{query} \PY{o}{=} \PY{l+s+s1}{\PYZsq{}\PYZsq{}\PYZsq{}}
          \PY{l+s+s1}{    SELECT Nama}
          \PY{l+s+s1}{    FROM karyawan}
          \PY{l+s+s1}{    WHERE Nomor\PYZus{}induk NOT IN (SELECT Nomor\PYZus{}induk FROM cuti\PYZus{}karyawan);}
          \PY{l+s+s1}{    }\PY{l+s+s1}{\PYZsq{}\PYZsq{}\PYZsq{}}
          \PY{n}{data} \PY{o}{=} \PY{n}{execute}\PY{p}{(}\PY{n}{query}\PY{p}{)}
          \PY{n+nb}{print}\PY{p}{(}\PY{l+s+s2}{\PYZdq{}}\PY{l+s+s2}{Karyawan yang tidak mengambil cuti adalah:}\PY{l+s+s2}{\PYZdq{}}\PY{p}{)}
          \PY{k}{for} \PY{n}{user} \PY{o+ow}{in} \PY{n}{data}\PY{p}{:}
              \PY{n+nb}{print}\PY{p}{(}\PY{l+s+s2}{\PYZdq{}}\PY{l+s+s2}{\PYZhy{}}\PY{l+s+s2}{\PYZdq{}}\PY{p}{,} \PY{n}{user}\PY{p}{[}\PY{l+m+mi}{0}\PY{p}{]}\PY{p}{)}
\end{Verbatim}


    \begin{Verbatim}[commandchars=\\\{\}]
Karyawan yang tidak mengambil cuti adalah:
- Amin
- Maulana
- Octavanus
- Nugroho
- Raisa

    \end{Verbatim}


    % Add a bibliography block to the postdoc
    
    
    
    \end{document}
